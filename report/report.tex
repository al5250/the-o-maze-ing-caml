\documentclass[11pt, margin=1in]{article}
\usepackage{amsmath}
\usepackage{amsfonts}
\usepackage[margin=1in]{geometry}
\usepackage{fancyhdr}
\newenvironment{proof}{\noindent {\it Proof.}}{\hfill\rule{2mm}{2mm}}
\pagestyle{fancy}
\lhead{\textbf{Math 23b: PSET 8}}
\rhead{\textit{Alex Lin}}
\cfoot{\thepage}
\renewcommand{\headrulewidth}{0.4pt}
\renewcommand{\footrulewidth}{0.4pt}
\makeatletter

%% NOTE: Use \texttt{} for code font.  

\begin{document}

\title{The O-Maze-ing Caml}
\author{Alex Lin and Melissa Yu}
\date{April 27, 2016}
\maketitle

% Brief summary here
 
The \textit{O-Maze-ing Caml} is an OCaml-based application that randomly generates mazes and computes the solutions to them.  The program also has graphical capabilities for rendering generated mazes onto the user's screen.  In designing this project, we intentionally employed recursive algorithms to take advantage of OCaml's functional paradigm.  The code can be found at https://github.com/al5250/the-o-maze-ing-caml.  

\section{High-Level Overview}

We begin with a high-level description of our project before delving into the specific details within the code files.  Section 2 address the Maze Generation portion of our program, while Section 3 focuses on Maze Solving.  

\subsection{Code Structure}  %% Brief description of main.ml, cell.ml, maze.ml and the code structure (the cell module, the maze functor, etc.)
The code is divided into three files:
\begin{itemize}
\item \texttt{main.ml} - executes the program 
\item \texttt{cell.ml}
\item \texttt{maze.ml}
\end{itemize} 

\subsection{Running the Program} %% Explain how to run the program using ./main.byte

\section{Maze Generation}

\subsection{Recursive-Division Algorithm}  %% Explain theory behind recursive division algorithm, how our algorithm executes it, and the design decisions we made (like the record type cell) 

\subsection{Functions Explained} %% Go through each function in generate and explain what it does/how they interact with each other

\subsection{Rendering Graphics} %% Describe how we display the solution to the maze using the Graphics module


\section{Maze Solving} 

\subsection{Recursive-Backtracking Algorithm} %% Explain theory behind recursive backtracking algorithm, how our algorithm executes it, and the design decisions we made

\subsection{Functions Explained} %% Go through each function in solve and explain what it does/how they interact with each other

\subsection{Rendering Graphics} %% Describe how we display the maze using the Graphics module after generating it





\end{document}